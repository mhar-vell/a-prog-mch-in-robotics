\section{Introdução}

Este documento e seu código-fonte são exemplos de referência de uso da classe
\textsf{abntex2} e do pacote \textsf{abntex2cite}. O documento exemplifica a
elaboração de publicação periódica científica impressa produzida conforme a ABNT
NBR 6022:2018 \emph{Informação e documentação - Artigo em publicação periódica
    científica - Apresentação}.

A expressão ``Modelo canônico'' é utilizada para indicar que \abnTeX\ não é
modelo específico de nenhuma universidade ou instituição, mas que implementa tão
somente os requisitos das normas da ABNT. Uma lista completa das normas
observadas pelo \abnTeX\ é apresentada em \citeonline{abntex2classe}.

Sinta-se convidado a participar do projeto \abnTeX! Acesse o site do projeto em
\url{http://www.abntex.net.br/}. Também fique livre para conhecer,
estudar, alterar e redistribuir o trabalho do \abnTeX, desde que os arquivos
modificados tenham seus nomes alterados e que os créditos sejam dados aos
autores originais, nos termos da ``The \LaTeX\ Project Public
License''\footnote{\url{http://www.latex-project.org/lppl.txt}}.

Encorajamos que sejam realizadas customizações específicas deste documento.
Porém, recomendamos que ao invés de se alterar diretamente os arquivos do
\abnTeX, distribua-se arquivos com as respectivas customizações. Isso permite
que futuras versões do \abnTeX~não se tornem automaticamente incompatíveis com
as customizações promovidas. Consulte \citeonline{abntex2-wiki-como-customizar}
para mais informações.

Este exemplo deve ser utilizado como complemento do manual da classe
\textsf{abntex2} \cite{abntex2classe}, dos manuais do pacote
\textsf{abntex2cite} \cite{abntex2cite,abntex2cite-alf} e do manual da classe
\textsf{memoir} \cite{memoir}. Consulte o \citeonline{abntex2modelo} para obter
exemplos e informações adicionais de uso de \abnTeX\ e de \LaTeX.

