\section{Mais exemplos no Modelo Canônico de Trabalhos Acadêmicos}

Este modelo de artigo é limitado em número de exemplos de comandos, pois são apresentados exclusivamente comandos diretamente relacionados com a produção de artigos.

Para exemplos adicionais de \abnTeX\ e \LaTeX, como inclusão de figuras,fórmulas matemáticas, citações, e outros, consulte o documento \citeonline{abntex2modelo}.

\section{Consulte o manual da classe \textsf{abntex2}}

Consulte o manual da classe \textsf{abntex2} \cite{abntex2classe} para uma referência completa das macros e ambientes disponíveis.

%----------------------------------------------------------------
\subsection{Estudo do estado da arte}
There have been published several works in which BNs are used for the diagnostics of machine failure. Among such dense selection of papers, few deal with machine failure prognostics using BNs as a probabilistic tool. However, in order to optimize our findings, the method BiLi was applied to refine our references and sources.

Firstly, during the naive cycle, different key-words were tested. A total of 32 trials were run in Scopus, in which the first trial was composed of the following key-words:



and the last string was the following:

\begin{center}
    \begin{tabular}{ c c c c c c c c c c }
        "" & failure & OR & fault & OR & fail & OR & breakdown & OR & degradation \\
        AND & prediction & OR & prognosis & OR & diagnosis \\
        AND & reliability & OR & maintenance systems & OR & condition based maintenaince \\
        AND & robotics & OR & robots \\
        AND NOT & bankrupcy & OR & bank & OR & patient & OR & forests

    \end{tabular}
\end{center}

 After the 32nd trial satisfied a annual growth rate of 18.56\% between 2015 and 2021, and demonstrated a co-citation network of four clusters, in which the two main ones demonstrated good connection, whereas the the two much smaller ones were isolated from each other, the key-words were refined through "litsearchr".The resulted key-words were the following:

 \begin{center}
    \begin{tabular}{ c c c c c c c c c c c c }
        "" & prognosis & OR & prediction \\
        AND & fault tree & OR & failure mode and effects analysis & OR & condition based maintenance \\
        AND & robotics system & OR & robot & OR & robotics & OR &equipment & OR & smart machine & OR & autonomous systems
    \end{tabular}
\end{center}

Beggining the optimized cylce, after computing the above key-words on scopus, a new .bib file was exported and analyzed in bibliometrix. Its annual scientific production was 11.61\%, and the co-citation network demonstrated four clusters, with two of them connected and the two others isolated by themselves. Therefore, it showed relevancy for the research.

Furthermore, in the impact cycle, after such file was ran through revtools, the most relevant publications were selected. This new .csv file was made into a .bib, which was then analyzed in bibliometrix. Finally, the five authors of highest H-index were noted using the "author impact by total citation" score. 

The production cycle was finalyzed through the reading and analysis of seven articles written by the following authors:
\begin{itemize}
    \item Wang Y \cite{wang2018}
    \item Liu Y \cite{liu2021}
    \item Robbersmyr KG \cite{duo2018} \cite{senanayaka2018robust}
    \item Van Khang H \cite{duo2018} \cite{kudelina2021methods}
    \item Wang F \cite{YANG201927}
    \item Wang J \cite{wangj2020}
\end{itemize}

Reading the articles cited above, a few conclusions were made. Firstly, this paper offers a different approach to calculating prognostic failures, based on tools like FTA, FMECA, and BNs, compared to the majority of articles from BiLi method. The sources selected were mainly focused on the usage of neural networks (NNs), deep learning, and machine learning. Furthermore, such NNs and algorithms demonstrated in the references were used to the purpose of diagnostics, detection, monitoring, and classification of machine failure in different types of systems. It can be said though that Wang F's paper \cite{YANG201927} does offer an attempt to predict remaining useful life (RUL) of a machine, however it does so by using a neural network which this research does not allign with.

\subsection{Application of FMECA and FTA}

\subsection{The development of the Bayesian Network}

\subsection{Robotics simulation}

%\subsection{Prognosis Machine package}

%\subsection{Prognosis Machine implementation}

%\subsection{Integrated simulation}
